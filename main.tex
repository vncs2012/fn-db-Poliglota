\documentclass[a4paper,12pt]{article}

\usepackage[utf8]{inputenc}
\usepackage[T1]{fontenc}
\usepackage{listings}
\usepackage{xcolor}
\usepackage{geometry}

\geometry{margin=1in}

\lstset{
    basicstyle=\ttfamily\small,
    breaklines=true,
    frame=single,
    numbers=left,
    numberstyle=\tiny,
    tabsize=2,
    showspaces=false,
    showstringspaces=false,
    keywordstyle=\color{blue},
    commentstyle=\color{gray},
    stringstyle=\color{red}
}

\lstdefinelanguage{SQL}{
    keywords={SELECT, FROM, WHERE, JOIN, ON, GROUP, BY, ORDER, LIMIT, COUNT, SUM, AVG},
    sensitive=false
}

\lstdefinelanguage{MongoDB}{
    keywords={db, find, aggregate, match, unwind, group, sort},
    sensitive=false
}

\lstdefinelanguage{Cypher}{
    keywords={MATCH, WHERE, RETURN, MERGE, SET, ORDER, BY, LIMIT},
    sensitive=false
}

\lstdefinelanguage{Redis}{
    keywords={HGETALL, ZREVRANGE, GET, INCR, SET},
    sensitive=false
}

\title{Consultas e Operações - DataDriven Store}
\author{Vinicius Carvalho Miranda}

\begin{document}

\maketitle

\section{Introdução}
Este documento apresenta um conjunto de consultas e operações para os bancos de dados utilizados na arquitetura da DataDriven Store. Cada banco de dados é utilizado para um propósito específico, aproveitando suas características únicas. As seções a seguir detalham cinco consultas ou operações para cada banco, incluindo PostgreSQL (relacional), MongoDB (NoSQL), Neo4j (grafos), ClickHouse (analítico) e Redis (cache).

\section{Docker \& Populando dados de exemplo}
Utilizando o Docker, podemos abrir o arquivo docker-compose.yml para verificar os serviços que serão iniciados com o comando. Abaixo estão os comandos necessários, que também incluem o processo de popular os dados de exemplo.

\subsection{Inicializando o Docker e populando tabelas}
\begin{lstlisting}[language=SQL]
docker compose up -d
\end{lstlisting}
\textbf{Uso:} Nesta parte, ele irá popular quase todas as tabelas, exceto o Neo4j, que precisaremos executar manualmente.

\subsection{Populando NEO4J}
\begin{lstlisting}[language=SQL]
docker exec -it datadriven_neo4j cypher-shell -u neo4j -p admin123 -a bolt://localhost:7687 -d neo4j -f /var/lib/neo4j/import/neo4j_schema.cypher
\end{lstlisting}

\section{PostgreSQL - Banco Relacional (OLTP)}
O PostgreSQL é usado para gerenciar dados transacionais estruturados, como clientes, pedidos e produtos.

\subsection{Consulta 1: Clientes com pedidos acima de um valor}
\begin{lstlisting}[language=SQL]
SELECT c.nome, c.email, p.id AS pedido_id, p.valor_total
FROM clientes c
JOIN pedidos p ON c.id = p.cliente_id
WHERE p.valor_total > 1000
ORDER BY p.valor_total DESC;
\end{lstlisting}
\textbf{Uso:} Identifica clientes que realizaram compras de alto valor.

\subsection{Consulta 2: Produtos com estoque baixo}
\begin{lstlisting}[language=SQL]
SELECT codigo_produto, nome, estoque, estoque_minimo
FROM produtos
WHERE estoque <= estoque_minimo AND ativo = TRUE
ORDER BY estoque ASC;
\end{lstlisting}
\textbf{Uso:} Auxilia na gestão de estoque.

\subsection{Consulta 3: Vendas por cidade}
\begin{lstlisting}[language=SQL]
SELECT c.cidade, COUNT(p.id) AS total_pedidos, SUM(p.valor_total) AS valor_total_vendas
FROM clientes c
JOIN pedidos p ON c.id = p.cliente_id
WHERE p.status = 'FINALIZADO'
GROUP BY c.cidade
ORDER BY valor_total_vendas DESC;
\end{lstlisting}
\textbf{Uso:} Fornece insights sobre desempenho de vendas por região.

\subsection{Consulta 4: Itens mais vendidos}
\begin{lstlisting}[language=SQL]
SELECT pr.nome, pr.categoria, SUM(ip.quantidade) AS quantidade_vendida, SUM(ip.subtotal) AS receita_total
FROM itens_pedido ip
JOIN produtos pr ON ip.produto_id = pr.id
JOIN pedidos p ON ip.pedido_id = p.id
WHERE p.data_pedido BETWEEN '2025-06-01' AND '2025-06-30'
GROUP BY pr.nome, pr.categoria
ORDER BY quantidade_vendida DESC
LIMIT 5;
\end{lstlisting}
\textbf{Uso:} Identifica produtos populares para planejamento.

\subsection{Consulta 5: Transações por método de pagamento}
\begin{lstlisting}[language=SQL]
SELECT metodo_pagamento, COUNT(id) AS total_transacoes, SUM(valor) AS valor_total
FROM transacoes_financeiras
WHERE status = 'APROVADO'
GROUP BY metodo_pagamento
ORDER BY valor_total DESC;
\end{lstlisting}
\textbf{Uso:} Analisa preferências de métodos de pagamento.

\section{MongoDB - Banco NoSQL (Documentos)}
O MongoDB armazena dados flexíveis, como detalhes de produtos e perfis de usuários.

\subsection{Consulta 1: Produtos por categoria e preço}
\begin{lstlisting}[language=MongoDB]
db.produtos.find(
    { categoria: "eletrônicos", preco: { $gte: 1000, $lte: 3000 } },
    { nome: 1, preco: 1, atributos: 1, _id: 0 }
).sort({ preco: 1 });
\end{lstlisting}
\textbf{Uso:} Filtra produtos para catálogos.

\subsection{Consulta 2: Produtos bem avaliados}
\begin{lstlisting}[language=MongoDB]
db.produtos.find(
    { "avaliacoes.nota": { $gte: 5 } },
    { nome: 1, avaliacoes: 1, _id: 0 }
);
\end{lstlisting}
\textbf{Uso:} Destaca produtos de alta qualidade.

\subsection{Consulta 3: Usuários com preferências específicas}
\begin{lstlisting}[language=MongoDB]
db.perfis_usuarios.find(
    { preferencias: "gaming" },
    { cliente_id: 1, preferencias: 1, produtos_favoritos: 1, _id: 0 }
);
\end{lstlisting}
\textbf{Uso:} Segmenta usuários para campanhas.

\subsection{Consulta 4: Nota média de avaliações}
\begin{lstlisting}[language=MongoDB]
db.produtos.aggregate([
    { $match: { _id: "NOTEBOOK001" } },
    { $unwind: "$avaliacoes" },
    { $group: { _id: "$_id", media_nota: { $avg: "$avaliacoes.nota" } } }
]);
\end{lstlisting}
\textbf{Uso:} Avalia a satisfação com um produto.

\subsection{Consulta 5: Produtos visualizados recentemente}
\begin{lstlisting}[language=MongoDB]
db.perfis_usuarios.find(
    { cliente_id: 1 },
    { historico_navegacao: 1, _id: 0 }
).sort({ "historico_navegacao.timestamp": -1 });
\end{lstlisting}
\textbf{Uso:} Personaliza a experiência do usuário.

\section{Neo4j - Banco de Grafos}
O Neo4j modela relacionamentos para sistemas de recomendação.

\subsection{Consulta 1: Recomendações baseadas em similaridade}
\begin{lstlisting}[language=Cypher]
MATCH (c1:Cliente {id: 1})-[:SIMILAR_A]->(c2:Cliente)-[:COMPROU]->(p:Produto)
WHERE NOT (c1)-[:COMPROU]->(p)
RETURN p.id, p.nome, COUNT(*) AS recomendacao_score
ORDER BY recomendacao_score DESC
LIMIT 5;
\end{lstlisting}
\textbf{Uso:} Gera recomendações personalizadas.

\subsection{Consulta 2: Produtos mais comprados por categoria}
\begin{lstlisting}[language=Cypher]
MATCH (c:Cliente)-[:COMPROU]->(p:Produto)-[:PERTENCE_A]->(cat:Categoria {nome: "eletrônicos"})
RETURN p.nome, COUNT(*) AS total_compras
ORDER BY total_compras DESC
LIMIT 5;
\end{lstlisting}
\textbf{Uso:} Identifica produtos populares.

\subsection{Consulta 3: Clientes que visualizaram sem comprar}
\begin{lstlisting}[language=Cypher]
MATCH (c:Cliente)-[:VISUALIZOU]->(p:Produto {id: "NOTEBOOK001"})
WHERE NOT (c)-[:COMPROU]->(p)
RETURN c.nome, c.email
ORDER BY c.nome;
\end{lstlisting}
\textbf{Uso:} Remarketing para clientes.

\subsection{Consulta 4: Marcas mais populares}
\begin{lstlisting}[language=Cypher]
MATCH (p:Produto)-[:DA_MARCA]->(m:Marca)<-[:DA_MARCA]-(p2:Produto)<-[:COMPROU]-(c:Cliente)
RETURN m.nome, COUNT(DISTINCT c) AS total_clientes
ORDER BY total_clientes DESC;
\end{lstlisting}
\textbf{Uso:} Analisa popularidade de marcas.

\subsection{Consulta 5: Caminhos de recomendação}
\begin{lstlisting}[language=Cypher]
MATCH path = (c:Cliente)-[:COMPROU]->(:Produto)-[:PERTENCE_A]->(:Categoria)-[:PERTENCE_A*0..1]->(:Categoria)
WHERE c.id = 1
RETURN path
LIMIT 10;
\end{lstlisting}
\textbf{Uso:} Explora conexões entre categorias.

\section{ClickHouse - Banco Analítico (OLAP)}
O ClickHouse processa eventos para análises rápidas.

\subsection{Consulta 1: Taxa de conversão por produto}
\begin{lstlisting}[language=SQL]
SELECT 
    produto_id,
    COUNT(*) AS total_visualizacoes,
    SUM(adicionou_carrinho) AS total_carrinho,
    SUM(comprou) AS total_compras,
    (SUM(comprou) * 100.0 / COUNT(*)) AS taxa_conversao
FROM funil_conversao
WHERE data >= '2025-06-01'
GROUP BY produto_id
ORDER BY taxa_conversao DESC;
\end{lstlisting}
\textbf{Uso:} Avalia eficiência de conversão.

\subsection{Consulta 2: Eventos por origem de tráfego}
\begin{lstlisting}[language=SQL]
SELECT 
    utm_source,
    COUNT(*) AS total_eventos,
    SUM(CASE WHEN evento_tipo = 'purchase' THEN 1 ELSE 0 END) AS total_compras,
    SUM(valor_total) AS receita_total
FROM eventos
WHERE data >= '2025-06-01'
GROUP BY utm_source
ORDER BY receita_total DESC;
\end{lstlisting}
\textbf{Uso:} Otimiza campanhas de marketing.

\subsection{Consulta 3: Produtos mais visualizados}
\begin{lstlisting}[language=SQL]
SELECT 
    produto_id,
    categoria,
    COUNT(*) AS total_visualizacoes
FROM eventos
WHERE evento_tipo = 'view' AND data BETWEEN '2025-06-01' AND '2025-06-30'
GROUP BY produto_id, categoria
ORDER BY total_visualizacoes DESC
LIMIT 10;
\end{lstlisting}
\textbf{Uso:} Ajusta catálogo com base em popularidade.

\subsection{Consulta 4: Comportamento por dispositivo}
\begin{lstlisting}[language=SQL]
SELECT 
    device_type,
    COUNT(DISTINCT session_id) AS sessoes_unicas,
    SUM(CASE WHEN evento_tipo = 'purchase' THEN valor_total ELSE 0 END) AS receita_total
FROM eventos
WHERE data >= '2025-06-01'
GROUP BY device_type
ORDER BY receita_total DESC;
\end{lstlisting}
\textbf{Uso:} Otimiza experiência por dispositivo.

\subsection{Consulta 5: Buscas populares}
\begin{lstlisting}[language=SQL]
SELECT 
    termo_busca,
    COUNT(*) AS total_buscas,
    AVG(resultados_busca) AS media_resultados
FROM eventos
WHERE evento_tipo = 'search' AND termo_busca != ''
GROUP BY termo_busca
ORDER BY total_buscas DESC
LIMIT 5;
\end{lstlisting}
\textbf{Uso:} Otimiza motor de busca.

\section{Redis - Cache e Sessões}
O Redis gerencia sessões, carrinhos e rankings em tempo real.

\subsection{Operação 1: Recuperar carrinho}
\begin{lstlisting}[language=Redis]
HGETALL cart:user_001
\end{lstlisting}
\textbf{Uso:} Exibe itens no carrinho.

\subsection{Operação 2: Produtos mais visualizados}
\begin{lstlisting}[language=Redis]
ZREVRANGE most_viewed 0 4 WITHSCORES
\end{lstlisting}
\textbf{Uso:} Destaca produtos populares.

\subsection{Operação 3: Verificar sessão}
\begin{lstlisting}[language=Redis]
GET session:user_001
\end{lstlisting}
\textbf{Uso:} Valida sessões ativas.

\subsection{Operação 4: Resultados de busca em cache}
\begin{lstlisting}[language=Redis]
GET search:gaming
\end{lstlisting}
\textbf{Uso:} Reduz latência em buscas.

\subsection{Operação 5: Contagem de visualizações}
\begin{lstlisting}[language=Redis]
INCR views:NOTEBOOK001:2025-06-28
\end{lstlisting}
\textbf{Uso:} Rastreia popularidade em tempo real.

\end{document}